\documentclass[preprint2]{aastex}
\usepackage[colorlinks]{hyperref}
\usepackage{amsmath, bm}
\usepackage{fancyhdr}
\usepackage{floatrow}
\usepackage[inner=2cm,outer=2cm,bottom=1cm,top=1.5cm]{geometry}
\usepackage{mathtools}
\usepackage{multicol}
\usepackage{verbatim}
\usepackage{graphicx,float,wrapfig, subcaption}
\usepackage{indentfirst}
\usepackage{soul}

\newcommand{\Ibar}{\mbox{\st{$I$}}}

\newcommand{\deri}[2]{\frac{\mathrm{d} #1}{\mathrm{d} #2}}
\newcommand{\inte}[4]{\int_{#1}^{#2} \! #3 \, \mathrm{d} #4}

\fancyhf{} % clear all header and footers
\renewcommand{\headrulewidth}{0pt} % remove the header rule
\rhead{\thepage}

%
%lfoot{\thepage} % puts it on the left side instead
%
% or if your document is 2 sided, and you want even and odd placement of the number
%facncyfoot[LE,RO]{\thepage} % Left side on Even pages; Right side on Odd pages
%
\pagestyle{fancy}

\begin{document}

\title{Differentially Private Interactive Media}

\author{\vspace{-0.5cm}{\hspace{0.1cm} \sc Jan Van Bruggen \hspace{1.6cm} James Chang \hspace{1.6cm} Cutter Coryell}} \email{jvanbrug@caltech.edu \hspace{1.2cm} jcchang@caltech.edu \hspace{1.2cm} ccoryell@caltech.edu}
\vspace{0.2cm}\affil{California Institute of Technology \vspace{1cm}}



\section{Introduction}

In the modern era, entities who own and sell digital content are increasingly concerned with unauthorized access to that content. A variety of ``digital rights management'' (DRM) technologies have been introduced to attempt to restrict content access to authorized users, with mixed results. In many cases arms races have resulted between content-providers and users in terms of rights management technologies and methods for breaking them. These technologies usually make it hard to share content between authorized users and unauthorized users and are commonly criticized for inconveniencing authorized users -- some content-providers push their lack of DRM as a selling point, and a general anti-DRM movement has sprung up.

One way to prevent unauthorized sharing of content is to change the nature of the content to make it fundamentally unsharable. For example, by \emph{streaming} content as opposed to handing out content in bundles which can be saved locally, it becomes non-trivial for the regular user to share their received content. Many music, movie/television, and even video-game services take this approach to control content access. However, it is not difficult for an unauthorized user to record the data stream in a way that can be saved locally, such that he manufactures his own content bundles which he can share. We say that the user has \emph{fully reconstructed} the content if she can provide an identical experience to an unauthorized user as an authorized user would receive through the official content channel.

It becomes significantly harder for a user to fully reconstruct the original content if the streamed content is in some way interactive. For example, in a streamed video-game, the data that is streamed to the user depends on the the input that the user provides. In order to provide the same experience to unauthorized users as the one that authorized users receive, the reconstructor must provide every possible time series of inputs to the content provider and record every resulting stream. If the content is deterministic -- the same time series of inputs always leads to the same data stream -- then this reconstruction ensures that any inputs that authorized users can give can also be given by unauthorized users \emph{with the same results}. This satisfies our definition of full reconstruction, because the experience of the unauthorized user is identical to that of the authorized user.\footnote{We define two users to have\emph{ identical experience} if, given the same time series of user inputs, the true probability distributions of certain content being streamed to each user is the same. This will be further motivated when we discuss noise in streaming.}

Yet another change that can be made to how content is provided which makes reconstruction more difficult is the addition of noise. This method is motivated by differential privacy, which is a notion of how the probability of a certain output to a query will change given a change in the queried database. By adding noise to a query in a special way, the particular existence or absence of a datum in the database can be concealed while still revealing useful information that respects differential privacy. The general idea of using noise to make it harder to reconstruct content is one we would like to borrow. With noise,\footnote{To be clear, when we refer to \emph{noise} we mean simply the introduction of randomness in the potential stream output -- far more general than the literal addition of noise values to the data.} every stream a content provider serves to a user is a random variable. Someone attempting to reconstruct the original content who, as before, only samples from the stream once (or once for every possible input time series if the content is interactive) will provide users with an empirical probability distribution where only one output has nonzero probability: that particular output the reconstructor sampled from the original content provider. This is quite possibly a poor approximation to the actual probability distribution governing possible outputs in the official stream.

How interactivity and noise make reconstruction harder, both individually and together, is an interesting question that we attempt to answer.

\section{Methods}

To investigate how the ease of reconstruction of a specific content-base depends on the interactiveness of the content stream and on the noise added to the content stream, we set up two simple models. In the first model, which lacks interaction, there is a space of possible outputs with utilities assigned by the content provider -- the particular output given to a user is chosen by the \emph{exponential mechanism} from the differential privacy literature.

In the second model, what is output is a \emph{sequence} of items. This model is Markovian in that the probability of any particular item appearing at a certain position in the sequence is purely a function of the previous item occurring in the sequence, and possibly user input related to this item. The first item of the sequence is chosen uniformly randomly, and subsequent items are chosen via the exponential mechanism. As in the first model, the content provider assigns utilities to each item-to-item transition which are used by the exponential mechanism. However, in this model we allow for interaction. At query-time, the user provides a vector of preferences, one for each item in the item space. The utilities of the transitions away from a particular item are allowed to be functions of the user preference value for that item.

We now develop the two models mathematically.

\subsection{Model 1}

Let \(\mathcal{S}\) be the set of all possible outputs to a streaming query. Let \(\mathcal{I} \subseteq \mathcal{S}\) be a set of \emph{intenteded} outputs -- these being, informally, the outputs that the content provider actually wishes the users to receive in the absence of noise. The content provider specifies \(\mathbf{u} = [(u_1, s_1), (u_2, s_2), (u_3, s_3), \dots]\), where each \(u_j \in [0, 1]\) and each \(s_j \in \mathcal{S}\). Here, \(\mathbf{u}\) is a multiset (elements can occur more than once) with any desired cardinality. In terms of differential privacy, \(\mathbf{u}\) is a database.



\section {Results and Discussion}

%\begin{figure}[H]
%\vspace{-0.24cm}
%\centering
%\hspace*{-1cm}\includegraphics[width=1.2\textwidth]{orbit_figs/radius.pdf}
%\caption{The radius \(r\), or orbital separation, of the Hulse-Taylor binary over a period of 20 hours. The oscillations in the separation are due to the ellipticity of the orbit.}
%\label{orbit-radius}
%\end{figure}

\section{Conclusions}

\section{Further Work}


%%%%%%%%%%%
\end{document}
